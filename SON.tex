%Papel A4, fonte Times tamanho 12
\documentclass[a4paper,12pt]{article}

% Determinando tamanhos de margens
\usepackage[left=2.5cm, right=2cm, top=3cm, bottom=3cm]{geometry}

% Preâmbulo para documentos em português
\usepackage[brazilian]{babel}     % Idioma do texto (regras de hifenização e textos automáticos i.e. Figura, Tabela)
\usepackage[utf8]{inputenc}       % Codificação do texto (caracteres especiais)
\usepackage[T1]{fontenc}          % Operações de fontes (tipo, tamanho, etc.)

% Pacote para inclusão de imagens PNG, JPEG e PDF
\usepackage[pdftex]{graphicx}
%\usepackage{subfigure}            % Pacote necessário para geração de subfiguras

% Cabecário com figura
\usepackage{fancyhdr}
\usepackage{tabularx}

\begin{document}

% Capa do relatório
\thispagestyle{empty}

\begin{flushleft}
\textbf{Sistema Meteorológico do Paraná --- SIMEPAR} \\
Curitiba --- PR, Caixa Postal 19100, CEP 81531-990, Tel/Fax: + 55 (41) 3320-2001
\end{flushleft}

\vspace{4.0cm}

\begin{center}
\fontsize{24pt}{28pt}\selectfont
\textbf{Relatório de acompanhamento climatológico na região da UHE Baixo Iguaçu}\\
\vspace{18pt}
\textbf{Setembro a Novembro de 2020}
\end{center}

\vspace{2.0cm}




\begin{center}
\textbf{José Eduardo Gonçalves}\\
Pesquisador\\

\vspace{12pt}

\textbf{Itamar Adilson Moreira}\\
Meteorologista\\

\vspace{12pt}

\textbf{André Luiz de Campos}\\
Engenheiro Civil\\
\end{center}

\vspace{36pt}

{\raggedleft \textbf{\textsc{Cesar Augustus Assis Beneti}}} \hfill \textsc{\textbf{Reinaldo Bonfim Silveira}}\\
\textit{Diretor Executivo} \hfill \textit{Coordenador de Modelagem Numérica}

\vspace{3.0cm}

\begin{center}
\textbf{Curitiba} \\
\textbf{Dezembro de 2020}
\end{center}

\newpage

% Sumário
\pagestyle{fancy}
\fancyhead{}     % Limpa cabeçalho
\fancyfoot{}     % Limpa rodapé
\fancyhead[L]{\includegraphics[width = 4cm]{../logo_simepar.png}}
\fancyhead[R]{\thepage}

\tableofcontents

\newpage

\listoffigures

\newpage

\listoftables


\newpage
\section{Introdução}

\hspace{0.5cm}
Em atendimento ao Projeto Básico Ambiental da Usina Hidrelétrica Baixo Iguaçu, o presente relatório tem por
finalidade apresentar dados monitorados e levantamentos estatísticos das variáveis meteorológicas da  
região de influêcia direta do empreendimento. Este acompanhameno é realizado trimestralmente,
sendo que essa análise corresponde ao trimestre setembro, outubro e novembro de 2019.
Conforme estabelecido no Projeto Básico Ambiental, são analisadas oito variáveis meteorológicas:

\begin{itemize}
\item temperatura do ar ($^\circ$C);
\item umidade relativa do ar (\%);
\item radiação solar incidente (MJm$^{-2}$dia$^{-1}$);
\item evapotranspiração  (mm/dia);
\item precipitação (mm);
\item pressão atmosférica (hPa);
\item velocidade do vento (m/s);
\item direição do vento (graus).
\end{itemize}

Vale ressaltar que a evapotranspiração diária é o único parâmetro que não foi medido diretamente. Neste caso optou-se por calcular a evapotranspiração de
referência com base no método de Penman-Monteith sugerido pela Organização das Nações Unidas para Agricultura e Alimentação (FAO) a partir dos demais
parâmetros meteorológicos medidos.

Além da estação do Simepar localizada na UHE Baixo Iguaçu, o monitoramento das variáveis meteorológicas na região é realizado em outras quatro
estações meteorológicas automáticas, duas operadas pelo SIMEPAR e duas pelo INMET. A frequência de aquisição dos dados nas estações
do INMET é horária, ao passo que os dados são registrados a cada 15 minutos nas estações do Simepar. Na sequência, a tabela \ref{tab:tabelaestacoes}
exibe as informações de cada estação.

\begin{table}[!hbt]
\centering
\caption{Informações dos postos de monitoramento na região da UHE Baixo Iguaçu.}
\begin{tabular}{lcccc}
\hline
Nome                           & Órgão Responsável & Latitude & Longitude & Início da Operação \\
\hline
\textbf{UHE Baixo Iguaçu}     & SIMEPAR           & -25,51   & -53,68    & 13/07/2017         \\
Cascavel                       & SIMEPAR           & -24,88   & -53,55    & 01/06/1997         \\
Dois Vizinhos                  & INMET             & -25,70   & -53,09    & 29/03/2007         \\
Planato                        & INMET             & -25,72   & -53,75    & 07/11/2007         \\
São Miguel do Iguaçu           & SIMEPAR           & -25,35   & -54,25    & 01/06/1997         \\
\hline
\end{tabular} \label{tab:tabelaestacoes}
\end{table}

Nas próximas seções serão apresentados os gráficos e as estatísticas dos parâmetros meteorológicos medidos nas cinco estações da região nos meses do trimestre analisado.
Destaca-se que as estatísticas foram calculadas apenas para meses com disponibilidade de pelo menos 65\% dos dados.

\newpage
\section{Setembro de 2020}

 
                   \subsection{Temperatura média do ar}
                   
                   \hspace{0.5cm} As séries de registros da temperatura média do ar diária do mês de setembro monitorada nas cinco estações da região da UHE Baixo Iguaçu são apresentadas na
                   figura \ref{fig:figtemp_med_ar_9}.    
    

    \begin{figure}[!htb]
    \includegraphics[width=1 \textwidth]{temp_med_ar_9}
    \caption{Temperatura média do ar observada na região da UHE Baixo Iguaçu em setembro de 2020.}
    \label{fig:figtemp_med_ar_9}
    \end{figure}

    A partir das séries diárias do mês em questão foram calculados o valor médio da temperatura média do ar, assim como seus valores máximo, mínimo e
    o desvio padrão em cada estação. Os resultados são apresentados na tabela \ref{tab:tabtemp_med_ar_9}.
    
    \begin{table}[!hbt]
    \begin{center}
    \caption{Estatísticas da variável temperatura média do ar ($^\circ$C) no mês de setembro de 2020 nas estações meteorológicas da região.}
    \label{tab:tabtemp_med_ar_9}
    \begin{tabular}{lccccc}
    \hline
                  &        Cascavel        &        Dois Vizinhos        &        Planalto        &         \begin{tabular}[c]{@{}c@{}} São Miguel \\ do Iguaçu \end{tabular}        &        \begin{tabular}[c]{@{}c@{}} UHE Baixo \\ Iguaçu \end{tabular}        \\
    \hline
    Média         &        22.79     &        22.02     &        23.65     &        24.27     &        \textbf{23.04}     \\
    Máxima        &        28.63     &        29.95     &        30.73     &        31.91     &        \textbf{29.66}     \\
    Mínima        &        17.27     &        15.00     &        19.60     &        18.46     &        \textbf{19.40}     \\
    Desvio-padrão &        3.00     &        3.45     &        3.10     &        4.06     &        \textbf{2.60}     \\
    \hline          
    \end{tabular}
    \end{center}
    \end{table}
    


    
    \newpage
     
                   \subsection{Temperatura máxima do ar}
                   
                   \hspace{0.5cm} As séries de registros da temperatura máxima do ar diária do mês de setembro monitorada nas cinco estações da região da UHE Baixo Iguaçu são apresentadas na
                   figura \ref{fig:figtemp_med_max_ar_9}.    
    

    \begin{figure}[!htb]
    \includegraphics[width=1 \textwidth]{temp_med_max_ar_9}
    \caption{Temperatura máxima do ar observada na região da UHE Baixo Iguaçu em setembro de 2020.}
    \label{fig:figtemp_med_max_ar_9}
    \end{figure}

    A partir das séries diárias do mês em questão foram calculados o valor médio da temperatura máxima do ar, assim como seus valores máximo, mínimo e
    o desvio padrão em cada estação. Os resultados são apresentados na tabela \ref{tab:tabtemp_med_max_ar_9}.
    
    \begin{table}[!hbt]
    \begin{center}
    \caption{Estatísticas da variável temperatura máxima do ar ($^\circ$C) no mês de setembro de 2020 nas estações meteorológicas da região.}
    \label{tab:tabtemp_med_max_ar_9}
    \begin{tabular}{lccccc}
    \hline
                  &        Cascavel        &        Dois Vizinhos        &        Planalto        &         \begin{tabular}[c]{@{}c@{}} São Miguel \\ do Iguaçu \end{tabular}        &        \begin{tabular}[c]{@{}c@{}} UHE Baixo \\ Iguaçu \end{tabular}        \\
    \hline
    Média         &        30.46     &        28.28     &        31.49     &        31.70     &        \textbf{33.02}     \\
    Máxima        &        37.70     &        35.40     &        40.20     &        40.60     &        \textbf{42.00}     \\
    Mínima        &        20.20     &        16.00     &        24.00     &        22.70     &        \textbf{23.50}     \\
    Desvio-padrão &        4.00     &        5.14     &        4.38     &        4.81     &        \textbf{4.90}     \\
    \hline          
    \end{tabular}
    \end{center}
    \end{table}
    


    
    \newpage
     
                   \subsection{Temperatura mínima do ar}
                   
                   \hspace{0.5cm} As séries de registros da temperatura mínima do ar diária do mês de setembro monitorada nas cinco estações da região da UHE Baixo Iguaçu são apresentadas na
                   figura \ref{fig:figtemp_med_min_ar_9}.    
    

    \begin{figure}[!htb]
    \includegraphics[width=1 \textwidth]{temp_med_min_ar_9}
    \caption{Temperatura mínima do ar observada na região da UHE Baixo Iguaçu em setembro de 2020.}
    \label{fig:figtemp_med_min_ar_9}
    \end{figure}

    A partir das séries diárias do mês em questão foram calculados o valor médio da temperatura mínima do ar, assim como seus valores máximo, mínimo e
    o desvio padrão em cada estação. Os resultados são apresentados na tabela \ref{tab:tabtemp_med_min_ar_9}.
    
    \begin{table}[!hbt]
    \begin{center}
    \caption{Estatísticas da variável temperatura mínima do ar ($^\circ$C) no mês de setembro de 2020 nas estações meteorológicas da região.}
    \label{tab:tabtemp_med_min_ar_9}
    \begin{tabular}{lccccc}
    \hline
                  &        Cascavel        &        Dois Vizinhos        &        Planalto        &         \begin{tabular}[c]{@{}c@{}} São Miguel \\ do Iguaçu \end{tabular}        &        \begin{tabular}[c]{@{}c@{}} UHE Baixo \\ Iguaçu \end{tabular}        \\
    \hline
    Média         &        16.34     &        16.94     &        17.48     &        17.02     &        \textbf{15.80}     \\
    Máxima        &        21.90     &        22.10     &        23.50     &        24.50     &        \textbf{21.30}     \\
    Mínima        &        11.50     &        13.00     &        13.00     &        9.50     &        \textbf{11.20}     \\
    Desvio-padrão &        2.66     &        2.20     &        2.69     &        3.78     &        \textbf{2.26}     \\
    \hline          
    \end{tabular}
    \end{center}
    \end{table}
    


    
    \newpage
     
                   \subsection{Umidade relativa}
                   
                   \hspace{0.5cm} As séries de registros da umidade relativa diária do mês de setembro monitorada nas cinco estações da região da UHE Baixo Iguaçu são apresentadas na
                   figura \ref{fig:figumidade_relativa_9}.    
    

    \begin{figure}[!htb]
    \includegraphics[width=1 \textwidth]{umidade_relativa_9}
    \caption{Umidade relativa observada na região da UHE Baixo Iguaçu em setembro de 2020.}
    \label{fig:figumidade_relativa_9}
    \end{figure}

    A partir das séries diárias do mês em questão foram calculados o valor médio da umidade relativa, assim como seus valores máximo, mínimo e
    o desvio padrão em cada estação. Os resultados são apresentados na tabela \ref{tab:tabumidade_relativa_9}.
    
    \begin{table}[!hbt]
    \begin{center}
    \caption{Estatísticas da variável umidade relativa (\%) no mês de setembro de 2020 nas estações meteorológicas da região.}
    \label{tab:tabumidade_relativa_9}
    \begin{tabular}{lccccc}
    \hline
                  &        Cascavel        &        Dois Vizinhos        &        Planalto        &         \begin{tabular}[c]{@{}c@{}} São Miguel \\ do Iguaçu \end{tabular}        &        \begin{tabular}[c]{@{}c@{}} UHE Baixo \\ Iguaçu \end{tabular}        \\
    \hline
    Média         &        56.10     &        62.52     &        57.08     &        56.96     &        \textbf{66.91}     \\
    Máxima        &        81.81     &        93.55     &        81.46     &        84.41     &        \textbf{82.98}     \\
    Mínima        &        35.50     &        34.60     &        35.22     &        28.62     &        \textbf{50.23}     \\
    Desvio-padrão &        11.77     &        12.49     &        13.00     &        13.59     &        \textbf{8.59}     \\
    \hline          
    \end{tabular}
    \end{center}
    \end{table}
    


    
    \newpage
     
                   \subsection{Pressão atmosférica}
                   
                   \hspace{0.5cm} As séries de registros da pressão atmosférica diária do mês de setembro monitorada nas cinco estações da região da UHE Baixo Iguaçu são apresentadas na
                   figura \ref{fig:figpressao_atm_9}.    
    

    \begin{figure}[!htb]
    \includegraphics[width=1 \textwidth]{pressao_atm_9}
    \caption{Pressão atmosférica observada na região da UHE Baixo Iguaçu em setembro de 2020.}
    \label{fig:figpressao_atm_9}
    \end{figure}

    A partir das séries diárias do mês em questão foram calculados o valor médio da pressão atmosférica, assim como seus valores máximo, mínimo e
    o desvio padrão em cada estação. Os resultados são apresentados na tabela \ref{tab:tabpressao_atm_9}.
    
    \begin{table}[!hbt]
    \begin{center}
    \caption{Estatísticas da variável pressão atmosférica (hPa) no mês de setembro de 2020 nas estações meteorológicas da região.}
    \label{tab:tabpressao_atm_9}
    \begin{tabular}{lccccc}
    \hline
                  &        Cascavel        &        Dois Vizinhos        &        Planalto        &         \begin{tabular}[c]{@{}c@{}} São Miguel \\ do Iguaçu \end{tabular}        &        \begin{tabular}[c]{@{}c@{}} UHE Baixo \\ Iguaçu \end{tabular}        \\
    \hline
    Média         &        937.45     &        952.14     &        967.64     &        979.07     &        \textbf{981.15}     \\
    Máxima        &        943.33     &        957.50     &        974.10     &        985.39     &        \textbf{987.75}     \\
    Mínima        &        930.96     &        949.37     &        959.77     &        971.03     &        \textbf{972.92}     \\
    Desvio-padrão &        2.57     &        2.25     &        3.05     &        3.20     &        \textbf{3.16}     \\
    \hline          
    \end{tabular}
    \end{center}
    \end{table}
    


    
    \newpage
     
                   \subsection{Radiação solar incidente}
                   
                   \hspace{0.5cm} As séries de registros da radiação solar incidente diária do mês de setembro monitorada nas cinco estações da região da UHE Baixo Iguaçu são apresentadas na
                   figura \ref{fig:figradiacao_9}.    
    

    \begin{figure}[!htb]
    \includegraphics[width=1 \textwidth]{radiacao_9}
    \caption{Radiação solar incidente observada na região da UHE Baixo Iguaçu em setembro de 2020.}
    \label{fig:figradiacao_9}
    \end{figure}

    A partir das séries diárias do mês em questão foram calculados o valor médio da radiação solar incidente, assim como seus valores máximo, mínimo e
    o desvio padrão em cada estação. Os resultados são apresentados na tabela \ref{tab:tabradiacao_9}.
    
    \begin{table}[!hbt]
    \begin{center}
    \caption{Estatísticas da variável radiação solar incidente (MJ m$^{-2}$ dia$^{-1}$) no mês de setembro de 2020 nas estações meteorológicas da região.}
    \label{tab:tabradiacao_9}
    \begin{tabular}{lccccc}
    \hline
                  &        Cascavel        &        Dois Vizinhos        &        Planalto        &         \begin{tabular}[c]{@{}c@{}} São Miguel \\ do Iguaçu \end{tabular}        &        \begin{tabular}[c]{@{}c@{}} UHE Baixo \\ Iguaçu \end{tabular}        \\
    \hline
    Média         &        17.89     &        10.32     &        14.75     &        16.02     &        \textbf{16.95}     \\
    Máxima        &        25.13     &        20.86     &        25.04     &        22.72     &        \textbf{24.40}     \\
    Mínima        &        2.39     &        1.26     &        2.55     &        5.25     &        \textbf{5.90}     \\
    Desvio-padrão &        5.21     &        6.50     &        8.10     &        4.83     &        \textbf{5.10}     \\
    \hline          
    \end{tabular}
    \end{center}
    \end{table}
    


    
    \newpage
     
                   \subsection{Evapotranspiração }
                   
                   \hspace{0.5cm} Uma vez que não há medição direta de evapotranspiração na região monitorada, optou-se por calcular a
                   evapotranspiração de referência com base no método de Penman-Monteith sugerido pela 
                   Organização das Nações Unidas para Agricultura e Alimentação (FAO). 
                   
                   Na sequência, a figura \ref{fig:figevapo_9} exibe os valores diários de evapotranspiração ao longo do mês de setembro nas cinco estações
                   monitoradas na área de interesse.
                       
    

    \begin{figure}[!htb]
    \includegraphics[width=1 \textwidth]{evapo_9}
    \caption{Evapotranspiração observada na região da UHE Baixo Iguaçu em setembro de 2020.}
    \label{fig:figevapo_9}
    \end{figure}

    A partir das séries diárias do mês em questão foram calculados o valor médio da evapotranspiração, assim como seus valores máximo, mínimo e
    o desvio padrão em cada estação. Os resultados são apresentados na tabela \ref{tab:tabevapo_9}.
    
    \begin{table}[!hbt]
    \begin{center}
    \caption{Estatísticas da variável evapotranspiração (mm/dia) no mês de setembro de 2020 nas estações meteorológicas da região.}
    \label{tab:tabevapo_9}
    \begin{tabular}{lccccc}
    \hline
                  &        Cascavel        &        Dois Vizinhos        &        Planalto        &         \begin{tabular}[c]{@{}c@{}} São Miguel \\ do Iguaçu \end{tabular}        &        \begin{tabular}[c]{@{}c@{}} UHE Baixo \\ Iguaçu \end{tabular}        \\
    \hline
    Média         &        5.87     &        2.80     &        4.40     &        4.62     &        \textbf{4.20}     \\
    Máxima        &        9.19     &        4.75     &        6.99     &        8.10     &        \textbf{6.37}     \\
    Mínima        &        1.70     &        0.92     &        1.30     &        1.39     &        \textbf{1.72}     \\
    Desvio-padrão &        1.80     &        1.20     &        1.66     &        1.57     &        \textbf{1.22}     \\
    \hline          
    \end{tabular}
    \end{center}
    \end{table}
    


    
    \newpage
     
                   \subsection{Velocidade}
                   
                   \hspace{0.5cm} As séries de registros da velocidade diária do mês de setembro monitorada nas cinco estações da região da UHE Baixo Iguaçu são apresentadas na
                   figura \ref{fig:figvelocidades_9}.    
    

    \begin{figure}[!htb]
    \includegraphics[width=1 \textwidth]{velocidades_9}
    \caption{Velocidade observada na região da UHE Baixo Iguaçu em setembro de 2020.}
    \label{fig:figvelocidades_9}
    \end{figure}

    A partir das séries diárias do mês em questão foram calculados o valor médio da velocidade, assim como seus valores máximo, mínimo e
    o desvio padrão em cada estação. Os resultados são apresentados na tabela \ref{tab:tabvelocidades_9}.
    
    \begin{table}[!hbt]
    \begin{center}
    \caption{Estatísticas da variável velocidade (m/s) no mês de setembro de 2020 nas estações meteorológicas da região.}
    \label{tab:tabvelocidades_9}
    \begin{tabular}{lccccc}
    \hline
                  &        Cascavel        &        Dois Vizinhos        &        Planalto        &         \begin{tabular}[c]{@{}c@{}} São Miguel \\ do Iguaçu \end{tabular}        &        \begin{tabular}[c]{@{}c@{}} UHE Baixo \\ Iguaçu \end{tabular}        \\
    \hline
    Média         &        4.66     &        1.29     &        2.18     &        2.56     &        \textbf{1.70}     \\
    Máxima        &        7.33     &        4.38     &        4.56     &        4.28     &        \textbf{2.92}     \\
    Mínima        &        2.21     &        0.00     &        0.09     &        0.87     &        \textbf{0.83}     \\
    Desvio-padrão &        1.37     &        1.20     &        1.01     &        0.77     &        \textbf{0.53}     \\
    \hline          
    \end{tabular}
    \end{center}
    \end{table}
    


    
    \newpage
      

                   \subsection{Direção do vento}
                   \hspace{0.5cm} A figura \ref{fig:figdir} exibe um diagrama com as direções predominantes da velocidade
                   do vento em cada dia dos mês. Além disso, para melhor visualização das condições do vento em cada estação, 
                   as linhas contínuas representam as séries de velocidade do vento normalizadas
                   pela velocidade máxima do mês registrada em cada estação.
    
    \begin{figure}[!htb]
    \includegraphics[width=1. \textwidth]{diagramaveldir_9}
    \caption{Diagrama da direção do vento predominante registrada nas estações meteorológicas na região da UHE Baixo Iguaçu.}
    \label{fig:figdir}
    \end{figure}
    
    A partir da série de dados de direção do vento no mês em questão foram contabilizadas as frequências em cada setor, conforme exibe a 
    tabela \ref{tab:tabdir}.

    \begin{table}[!hbt]
    \begin{center}
    \caption{Frequência (\%) das direções do vento medidos nas estações meteorológicas no mês de setembro de 2020.}
    \label{tab:tabdir}
    \begin{tabular}{lcccccccc}
    \hline
             &        N      &        NE     &        L      &        SE     &        S     &        SO     &        O    &        NO  \\
    \hline                                                                                                                            
    Cascavel       &        3.33     &        56.67     &        6.67     &        20.00     &        0.00    &        0.00     &        6.67   &        6.67   \\
    Dois Vizinhos       &        3.85     &        3.85     &        0.00     &        11.54     &        53.85    &        19.23     &        0.00   &        7.69   \\
    Planalto       &        3.33     &        6.67     &        3.33     &        0.00     &        0.00    &        73.33     &        6.67   &        6.67   \\
    São Miguel do Iguaçu       &        13.33     &        6.67     &        6.67     &        43.33     &        26.67    &        0.00     &        0.00   &        3.33   \\
    \textbf{UHE Baixo Iguacu}       &        \textbf{3.33}     &        \textbf{13.33}     &        \textbf{53.33}     &        \textbf{3.33}     &        \textbf{3.33}    &        \textbf{6.67}     &        \textbf{13.33}   &        \textbf{3.33}   \\

    \hline          
    \end{tabular}
    \end{center}
    \end{table}

\newpage


                   \subsection{Precipitação}
                   \hspace{0.5cm} As séries de registros de precipitação diária nas estações telemétricas da região da UHE Baixo Iguaçu são exibidas na figura \ref{fig:figchuva}, e a
                   tabela \ref{tab:tabchuva} exibe o acumulado mensal de cada estação.

    \begin{figure}[!htb]
    \includegraphics[width=1 \textwidth]{chuva_9}
    \caption{Registros de chuva acumulada diária observada na região da UHE Baixo Iguaçu ao longo de setembro de 2020.}
    \label{fig:figchuva}
    \end{figure}
    
    Na sequência, a tabela \ref{tab:tabchuva} exibe os valores da chuva acumulada mensal em cada estação, 
    assim como o valor do maior registro de chuva diária.
    
    \begin{table}[!htb]
    \centering
    \caption{Chuva acumulada (mm) nas estações telemétricas próximas à região da UHE Baixo Iguaçu ao longo de setembro de 2020.}
    \label{tab:tabchuva}
    \begin{tabular}{lcc}
    \hline
                           & Chuva diária máxima      &   Chuva acumulada mensal \\
    \hline
      Cascavel                   &  11.40                   &   20.20    \\
      Dois Vizinhos                   &  19.20                   &   19.60    \\
      Planalto                   &  15.40                   &   25.60    \\
      São Miguel do Iguaçu                   &  22.80                   &   33.20    \\
      \textbf{UHE Baixo Iguacu}                   &  \textbf{27.00}                   &   \textbf{38.00}     \\
    \hline
    \end{tabular}
    \end{table}
        
\newpage
\section{Outubro de 2020}

                   \subsection{Temperatura média do ar}
                   
                   \hspace{0.5cm} As séries de registros da temperatura média do ar diária do mês de outubro monitorada nas cinco estações da região da UHE Baixo Iguaçu são apresentadas na
                   figura \ref{fig:figtemp_med_ar_10}.    
    

    \begin{figure}[!htb]
    \includegraphics[width=1 \textwidth]{temp_med_ar_10}
    \caption{Temperatura média do ar observada na região da UHE Baixo Iguaçu em outubro de 2020.}
    \label{fig:figtemp_med_ar_10}
    \end{figure}

    A partir das séries diárias do mês em questão foram calculados o valor médio da temperatura média do ar, assim como seus valores máximo, mínimo e
    o desvio padrão em cada estação. Os resultados são apresentados na tabela \ref{tab:tabtemp_med_ar_10}.
    
    \begin{table}[!hbt]
    \begin{center}
    \caption{Estatísticas da variável temperatura média do ar ($^\circ$C) no mês de outubro de 2020 nas estações meteorológicas da região.}
    \label{tab:tabtemp_med_ar_10}
    \begin{tabular}{lccccc}
    \hline
                  &        Cascavel        &        Dois Vizinhos        &        Planalto        &         \begin{tabular}[c]{@{}c@{}} São Miguel \\ do Iguaçu \end{tabular}        &        \begin{tabular}[c]{@{}c@{}} UHE Baixo \\ Iguaçu \end{tabular}        \\
    \hline
    Média         &        23.63     &        27.94     &        24.78     &        25.77     &        \textbf{24.33}     \\
    Máxima        &        31.09     &        34.06     &        31.18     &        33.20     &        \textbf{28.46}     \\
    Mínima        &        19.30     &        22.50     &        19.68     &        20.78     &        \textbf{19.84}     \\
    Desvio-padrão &        3.35     &        3.60     &        2.87     &        3.01     &        \textbf{2.57}     \\
    \hline          
    \end{tabular}
    \end{center}
    \end{table}
    


    
    \newpage
     
                   \subsection{Temperatura máxima do ar}
                   
                   \hspace{0.5cm} As séries de registros da temperatura máxima do ar diária do mês de outubro monitorada nas cinco estações da região da UHE Baixo Iguaçu são apresentadas na
                   figura \ref{fig:figtemp_med_max_ar_10}.    
    

    \begin{figure}[!htb]
    \includegraphics[width=1 \textwidth]{temp_med_max_ar_10}
    \caption{Temperatura máxima do ar observada na região da UHE Baixo Iguaçu em outubro de 2020.}
    \label{fig:figtemp_med_max_ar_10}
    \end{figure}

    A partir das séries diárias do mês em questão foram calculados o valor médio da temperatura máxima do ar, assim como seus valores máximo, mínimo e
    o desvio padrão em cada estação. Os resultados são apresentados na tabela \ref{tab:tabtemp_med_max_ar_10}.
    
    \begin{table}[!hbt]
    \begin{center}
    \caption{Estatísticas da variável temperatura máxima do ar ($^\circ$C) no mês de outubro de 2020 nas estações meteorológicas da região.}
    \label{tab:tabtemp_med_max_ar_10}
    \begin{tabular}{lccccc}
    \hline
                  &        Cascavel        &        Dois Vizinhos        &        Planalto        &         \begin{tabular}[c]{@{}c@{}} São Miguel \\ do Iguaçu \end{tabular}        &        \begin{tabular}[c]{@{}c@{}} UHE Baixo \\ Iguaçu \end{tabular}        \\
    \hline
    Média         &        31.57     &        30.57     &        32.32     &        32.92     &        \textbf{33.61}     \\
    Máxima        &        38.60     &        38.00     &        39.40     &        40.00     &        \textbf{40.90}     \\
    Mínima        &        22.90     &        24.20     &        24.90     &        25.50     &        \textbf{24.60}     \\
    Desvio-padrão &        4.30     &        3.42     &        3.84     &        3.78     &        \textbf{3.91}     \\
    \hline          
    \end{tabular}
    \end{center}
    \end{table}
    


    
    \newpage
     
                   \subsection{Temperatura mínima do ar}
                   
                   \hspace{0.5cm} As séries de registros da temperatura mínima do ar diária do mês de outubro monitorada nas cinco estações da região da UHE Baixo Iguaçu são apresentadas na
                   figura \ref{fig:figtemp_med_min_ar_10}.    
    

    \begin{figure}[!htb]
    \includegraphics[width=1 \textwidth]{temp_med_min_ar_10}
    \caption{Temperatura mínima do ar observada na região da UHE Baixo Iguaçu em outubro de 2020.}
    \label{fig:figtemp_med_min_ar_10}
    \end{figure}

    A partir das séries diárias do mês em questão foram calculados o valor médio da temperatura mínima do ar, assim como seus valores máximo, mínimo e
    o desvio padrão em cada estação. Os resultados são apresentados na tabela \ref{tab:tabtemp_med_min_ar_10}.
    
    \begin{table}[!hbt]
    \begin{center}
    \caption{Estatísticas da variável temperatura mínima do ar ($^\circ$C) no mês de outubro de 2020 nas estações meteorológicas da região.}
    \label{tab:tabtemp_med_min_ar_10}
    \begin{tabular}{lccccc}
    \hline
                  &        Cascavel        &        Dois Vizinhos        &        Planalto        &         \begin{tabular}[c]{@{}c@{}} São Miguel \\ do Iguaçu \end{tabular}        &        \begin{tabular}[c]{@{}c@{}} UHE Baixo \\ Iguaçu \end{tabular}        \\
    \hline
    Média         &        16.93     &        24.55     &        18.35     &        18.77     &        \textbf{17.36}     \\
    Máxima        &        24.60     &        31.50     &        23.60     &        27.60     &        \textbf{20.50}     \\
    Mínima        &        10.60     &        18.20     &        12.00     &        12.80     &        \textbf{11.20}     \\
    Desvio-padrão &        3.19     &        4.21     &        2.96     &        3.43     &        \textbf{2.61}     \\
    \hline          
    \end{tabular}
    \end{center}
    \end{table}
    


    
    \newpage
     
                   \subsection{Umidade relativa}
                   
                   \hspace{0.5cm} As séries de registros da umidade relativa diária do mês de outubro monitorada nas cinco estações da região da UHE Baixo Iguaçu são apresentadas na
                   figura \ref{fig:figumidade_relativa_10}.    
    

    \begin{figure}[!htb]
    \includegraphics[width=1 \textwidth]{umidade_relativa_10}
    \caption{Umidade relativa observada na região da UHE Baixo Iguaçu em outubro de 2020.}
    \label{fig:figumidade_relativa_10}
    \end{figure}

    A partir das séries diárias do mês em questão foram calculados o valor médio da umidade relativa, assim como seus valores máximo, mínimo e
    o desvio padrão em cada estação. Os resultados são apresentados na tabela \ref{tab:tabumidade_relativa_10}.
    
    \begin{table}[!hbt]
    \begin{center}
    \caption{Estatísticas da variável umidade relativa (\%) no mês de outubro de 2020 nas estações meteorológicas da região.}
    \label{tab:tabumidade_relativa_10}
    \begin{tabular}{lccccc}
    \hline
                  &        Cascavel        &        Dois Vizinhos        &        Planalto        &         \begin{tabular}[c]{@{}c@{}} São Miguel \\ do Iguaçu \end{tabular}        &        \begin{tabular}[c]{@{}c@{}} UHE Baixo \\ Iguaçu \end{tabular}        \\
    \hline
    Média         &        56.75     &        44.00     &        56.30     &        56.51     &        \textbf{66.24}     \\
    Máxima        &        81.48     &        75.20     &        80.50     &        75.58     &        \textbf{79.22}     \\
    Mínima        &        34.98     &        27.20     &        38.29     &        35.24     &        \textbf{54.84}     \\
    Desvio-padrão &        10.96     &        13.50     &        10.97     &        9.93     &        \textbf{6.85}     \\
    \hline          
    \end{tabular}
    \end{center}
    \end{table}
    


    
    \newpage
     
                   \subsection{Pressão atmosférica}
                   
                   \hspace{0.5cm} As séries de registros da pressão atmosférica diária do mês de outubro monitorada nas cinco estações da região da UHE Baixo Iguaçu são apresentadas na
                   figura \ref{fig:figpressao_atm_10}.    
    

    \begin{figure}[!htb]
    \includegraphics[width=1 \textwidth]{pressao_atm_10}
    \caption{Pressão atmosférica observada na região da UHE Baixo Iguaçu em outubro de 2020.}
    \label{fig:figpressao_atm_10}
    \end{figure}

    A partir das séries diárias do mês em questão foram calculados o valor médio da pressão atmosférica, assim como seus valores máximo, mínimo e
    o desvio padrão em cada estação. Os resultados são apresentados na tabela \ref{tab:tabpressao_atm_10}.
    
    \begin{table}[!hbt]
    \begin{center}
    \caption{Estatísticas da variável pressão atmosférica (hPa) no mês de outubro de 2020 nas estações meteorológicas da região.}
    \label{tab:tabpressao_atm_10}
    \begin{tabular}{lccccc}
    \hline
                  &        Cascavel        &        Dois Vizinhos        &        Planalto        &         \begin{tabular}[c]{@{}c@{}} São Miguel \\ do Iguaçu \end{tabular}        &        \begin{tabular}[c]{@{}c@{}} UHE Baixo \\ Iguaçu \end{tabular}        \\
    \hline
    Média         &        935.71     &        949.73     &        966.00     &        977.08     &        \textbf{979.44}     \\
    Máxima        &        940.70     &        954.66     &        971.23     &        982.33     &        \textbf{984.84}     \\
    Mínima        &        930.71     &        943.58     &        960.91     &        971.89     &        \textbf{974.56}     \\
    Desvio-padrão &        2.49     &        3.03     &        2.77     &        2.82     &        \textbf{2.85}     \\
    \hline          
    \end{tabular}
    \end{center}
    \end{table}
    


    
    \newpage
     
                   \subsection{Radiação solar incidente}
                   
                   \hspace{0.5cm} As séries de registros da radiação solar incidente diária do mês de outubro monitorada nas cinco estações da região da UHE Baixo Iguaçu são apresentadas na
                   figura \ref{fig:figradiacao_10}.    
    

    \begin{figure}[!htb]
    \includegraphics[width=1 \textwidth]{radiacao_10}
    \caption{Radiação solar incidente observada na região da UHE Baixo Iguaçu em outubro de 2020.}
    \label{fig:figradiacao_10}
    \end{figure}

    A partir das séries diárias do mês em questão foram calculados o valor médio da radiação solar incidente, assim como seus valores máximo, mínimo e
    o desvio padrão em cada estação. Os resultados são apresentados na tabela \ref{tab:tabradiacao_10}.
    
    \begin{table}[!hbt]
    \begin{center}
    \caption{Estatísticas da variável radiação solar incidente (MJ m$^{-2}$ dia$^{-1}$) no mês de outubro de 2020 nas estações meteorológicas da região.}
    \label{tab:tabradiacao_10}
    \begin{tabular}{lccccc}
    \hline
                  &        Cascavel        &        Dois Vizinhos        &        Planalto        &         \begin{tabular}[c]{@{}c@{}} São Miguel \\ do Iguaçu \end{tabular}        &        \begin{tabular}[c]{@{}c@{}} UHE Baixo \\ Iguaçu \end{tabular}        \\
    \hline
    Média         &        20.36     &        ---     &        22.09     &        20.11     &        \textbf{21.72}     \\
    Máxima        &        30.06     &        ---     &        30.04     &        27.77     &        \textbf{29.96}     \\
    Mínima        &        6.43     &        ---     &        6.69     &        6.62     &        \textbf{5.93}     \\
    Desvio-padrão &        5.69     &        ---     &        6.07     &        5.30     &        \textbf{6.11}     \\
    \hline          
    \end{tabular}
    \end{center}
    \end{table}
    


    
    \newpage
     
                   \subsection{Evapotranspiração }
                   
                   \hspace{0.5cm} Uma vez que não há medição direta de evapotranspiração na região monitorada, optou-se por calcular a
                   evapotranspiração de referência com base no método de Penman-Monteith sugerido pela 
                   Organização das Nações Unidas para Agricultura e Alimentação (FAO). 
                   
                   Na sequência, a figura \ref{fig:figevapo_10} exibe os valores diários de evapotranspiração ao longo do mês de outubro nas cinco estações
                   monitoradas na área de interesse.
                       
    

    \begin{figure}[!htb]
    \includegraphics[width=1 \textwidth]{evapo_10}
    \caption{Evapotranspiração observada na região da UHE Baixo Iguaçu em outubro de 2020.}
    \label{fig:figevapo_10}
    \end{figure}

    A partir das séries diárias do mês em questão foram calculados o valor médio da evapotranspiração, assim como seus valores máximo, mínimo e
    o desvio padrão em cada estação. Os resultados são apresentados na tabela \ref{tab:tabevapo_10}.
    
    \begin{table}[!hbt]
    \begin{center}
    \caption{Estatísticas da variável evapotranspiração (mm/dia) no mês de outubro de 2020 nas estações meteorológicas da região.}
    \label{tab:tabevapo_10}
    \begin{tabular}{lccccc}
    \hline
                  &        Cascavel        &        Dois Vizinhos        &        Planalto        &         \begin{tabular}[c]{@{}c@{}} São Miguel \\ do Iguaçu \end{tabular}        &        \begin{tabular}[c]{@{}c@{}} UHE Baixo \\ Iguaçu \end{tabular}        \\
    \hline
    Média         &        6.16     &        ---     &        5.49     &        5.28     &        \textbf{5.11}     \\
    Máxima        &        8.38     &        ---     &        8.08     &        6.94     &        \textbf{6.99}     \\
    Mínima        &        2.88     &        ---     &        2.16     &        2.59     &        \textbf{1.90}     \\
    Desvio-padrão &        1.48     &        ---     &        1.43     &        1.11     &        \textbf{1.22}     \\
    \hline          
    \end{tabular}
    \end{center}
    \end{table}
    


    
    \newpage
     
                   \subsection{Velocidade}
                   
                   \hspace{0.5cm} As séries de registros da velocidade diária do mês de outubro monitorada nas cinco estações da região da UHE Baixo Iguaçu são apresentadas na
                   figura \ref{fig:figvelocidades_10}.    
    

    \begin{figure}[!htb]
    \includegraphics[width=1 \textwidth]{velocidades_10}
    \caption{Velocidade observada na região da UHE Baixo Iguaçu em outubro de 2020.}
    \label{fig:figvelocidades_10}
    \end{figure}

    A partir das séries diárias do mês em questão foram calculados o valor médio da velocidade, assim como seus valores máximo, mínimo e
    o desvio padrão em cada estação. Os resultados são apresentados na tabela \ref{tab:tabvelocidades_10}.
    
    \begin{table}[!hbt]
    \begin{center}
    \caption{Estatísticas da variável velocidade (m/s) no mês de outubro de 2020 nas estações meteorológicas da região.}
    \label{tab:tabvelocidades_10}
    \begin{tabular}{lccccc}
    \hline
                  &        Cascavel        &        Dois Vizinhos        &        Planalto        &         \begin{tabular}[c]{@{}c@{}} São Miguel \\ do Iguaçu \end{tabular}        &        \begin{tabular}[c]{@{}c@{}} UHE Baixo \\ Iguaçu \end{tabular}        \\
    \hline
    Média         &        4.19     &        1.21     &        2.34     &        2.40     &        \textbf{1.84}     \\
    Máxima        &        5.84     &        5.10     &        4.15     &        3.74     &        \textbf{3.05}     \\
    Mínima        &        2.33     &        0.00     &        0.49     &        1.46     &        \textbf{0.94}     \\
    Desvio-padrão &        0.80     &        1.10     &        0.99     &        0.51     &        \textbf{0.57}     \\
    \hline          
    \end{tabular}
    \end{center}
    \end{table}
    


    
    \newpage
      

                   \subsection{Direção do vento}
                   \hspace{0.5cm} A figura \ref{fig:figdir} exibe um diagrama com as direções predominantes da velocidade
                   do vento em cada dia dos mês. Além disso, para melhor visualização das condições do vento em cada estação, 
                   as linhas contínuas representam as séries de velocidade do vento normalizadas
                   pela velocidade máxima do mês registrada em cada estação.
    
    \begin{figure}[!htb]
    \includegraphics[width=1. \textwidth]{diagramaveldir_10}
    \caption{Diagrama da direção do vento predominante registrada nas estações meteorológicas na região da UHE Baixo Iguaçu.}
    \label{fig:figdir}
    \end{figure}
    
    A partir da série de dados de direção do vento no mês em questão foram contabilizadas as frequências em cada setor, conforme exibe a 
    tabela \ref{tab:tabdir}.

    \begin{table}[!hbt]
    \begin{center}
    \caption{Frequência (\%) das direções do vento medidos nas estações meteorológicas no mês de outubro de 2020.}
    \label{tab:tabdir}
    \begin{tabular}{lcccccccc}
    \hline
             &        N      &        NE     &        L      &        SE     &        S     &        SO     &        O    &        NO  \\
    \hline                                                                                                                            
    Cascavel       &        0.00     &        58.06     &        19.35     &        16.13     &        3.23    &        0.00     &        0.00   &        3.23   \\
    Dois Vizinhos       &        26.09     &        4.35     &        0.00     &        17.39     &        13.04    &        26.09     &        0.00   &        13.04   \\
    Planalto       &        6.45     &        3.23     &        3.23     &        0.00     &        0.00    &        77.42     &        9.68   &        0.00   \\
    São Miguel do Iguaçu       &        0.00     &        9.68     &        12.90     &        61.29     &        16.13    &        0.00     &        0.00   &        0.00   \\
    \textbf{UHE Baixo Iguacu}       &        \textbf{3.23}     &        \textbf{16.13}     &        \textbf{61.29}     &        \textbf{0.00}     &        \textbf{3.23}    &        \textbf{0.00}     &        \textbf{12.90}   &        \textbf{3.23}   \\

    \hline          
    \end{tabular}
    \end{center}
    \end{table}

\newpage


                   \subsection{Precipitação}
                   \hspace{0.5cm} As séries de registros de precipitação diária nas estações telemétricas da região da UHE Baixo Iguaçu são exibidas na figura \ref{fig:figchuva}, e a
                   tabela \ref{tab:tabchuva} exibe o acumulado mensal de cada estação.

    \begin{figure}[!htb]
    \includegraphics[width=1 \textwidth]{chuva_10}
    \caption{Registros de chuva acumulada diária observada na região da UHE Baixo Iguaçu ao longo de outubro de 2020.}
    \label{fig:figchuva}
    \end{figure}
    
    Na sequência, a tabela \ref{tab:tabchuva} exibe os valores da chuva acumulada mensal em cada estação, 
    assim como o valor do maior registro de chuva diária.
    
    \begin{table}[!htb]
    \centering
    \caption{Chuva acumulada (mm) nas estações telemétricas próximas à região da UHE Baixo Iguaçu ao longo de outubro de 2020.}
    \label{tab:tabchuva}
    \begin{tabular}{lcc}
    \hline
                           & Chuva diária máxima      &   Chuva acumulada mensal \\
    \hline
      Cascavel                   &  23.40                   &   33.40    \\
      Dois Vizinhos                   &  0.00                   &   0.00    \\
      Planalto                   &  24.80                   &   72.00    \\
      São Miguel do Iguaçu                   &  36.20                   &   82.40    \\
      \textbf{UHE Baixo Iguacu}                   &  \textbf{24.20}                   &   \textbf{65.00}     \\
    \hline
    \end{tabular}
    \end{table}
        

\newpage
\section{Novembro de 2020}
                   \subsection{Temperatura média do ar}
                   
                   \hspace{0.5cm} As séries de registros da temperatura média do ar diária do mês de novembro monitorada nas cinco estações da região da UHE Baixo Iguaçu são apresentadas na
                   figura \ref{fig:figtemp_med_ar_11}.    
    

    \begin{figure}[!htb]
    \includegraphics[width=1 \textwidth]{temp_med_ar_11}
    \caption{Temperatura média do ar observada na região da UHE Baixo Iguaçu em novembro de 2020.}
    \label{fig:figtemp_med_ar_11}
    \end{figure}

    A partir das séries diárias do mês em questão foram calculados o valor médio da temperatura média do ar, assim como seus valores máximo, mínimo e
    o desvio padrão em cada estação. Os resultados são apresentados na tabela \ref{tab:tabtemp_med_ar_11}.
    
    \begin{table}[!hbt]
    \begin{center}
    \caption{Estatísticas da variável temperatura média do ar ($^\circ$C) no mês de novembro de 2020 nas estações meteorológicas da região.}
    \label{tab:tabtemp_med_ar_11}
    \begin{tabular}{lccccc}
    \hline
                  &        Cascavel        &        Dois Vizinhos        &        Planalto        &         \begin{tabular}[c]{@{}c@{}} São Miguel \\ do Iguaçu \end{tabular}        &        \begin{tabular}[c]{@{}c@{}} UHE Baixo \\ Iguaçu \end{tabular}        \\
    \hline
    Média         &        22.32     &        26.97     &        24.47     &        25.24     &        \textbf{23.91}     \\
    Máxima        &        27.22     &        32.58     &        30.62     &        31.80     &        \textbf{29.45}     \\
    Mínima        &        16.58     &        19.75     &        18.64     &        18.51     &        \textbf{17.81}     \\
    Desvio-padrão &        2.38     &        2.79     &        2.68     &        2.87     &        \textbf{2.39}     \\
    \hline          
    \end{tabular}
    \end{center}
    \end{table}
    


    
    \newpage
     
                   \subsection{Temperatura máxima do ar}
                   
                   \hspace{0.5cm} As séries de registros da temperatura máxima do ar diária do mês de novembro monitorada nas cinco estações da região da UHE Baixo Iguaçu são apresentadas na
                   figura \ref{fig:figtemp_med_max_ar_11}.    
    

    \begin{figure}[!htb]
    \includegraphics[width=1 \textwidth]{temp_med_max_ar_11}
    \caption{Temperatura máxima do ar observada na região da UHE Baixo Iguaçu em novembro de 2020.}
    \label{fig:figtemp_med_max_ar_11}
    \end{figure}

    A partir das séries diárias do mês em questão foram calculados o valor médio da temperatura máxima do ar, assim como seus valores máximo, mínimo e
    o desvio padrão em cada estação. Os resultados são apresentados na tabela \ref{tab:tabtemp_med_max_ar_11}.
    
    \begin{table}[!hbt]
    \begin{center}
    \caption{Estatísticas da variável temperatura máxima do ar ($^\circ$C) no mês de novembro de 2020 nas estações meteorológicas da região.}
    \label{tab:tabtemp_med_max_ar_11}
    \begin{tabular}{lccccc}
    \hline
                  &        Cascavel        &        Dois Vizinhos        &        Planalto        &         \begin{tabular}[c]{@{}c@{}} São Miguel \\ do Iguaçu \end{tabular}        &        \begin{tabular}[c]{@{}c@{}} UHE Baixo \\ Iguaçu \end{tabular}        \\
    \hline
    Média         &        29.72     &        29.74     &        32.17     &        32.33     &        \textbf{33.56}     \\
    Máxima        &        34.50     &        35.40     &        37.70     &        38.60     &        \textbf{39.60}     \\
    Mínima        &        20.30     &        20.00     &        22.10     &        22.20     &        \textbf{21.50}     \\
    Desvio-padrão &        3.25     &        3.04     &        3.59     &        3.82     &        \textbf{4.30}     \\
    \hline          
    \end{tabular}
    \end{center}
    \end{table}
    


    
    \newpage
     
                   \subsection{Temperatura mínima do ar}
                   
                   \hspace{0.5cm} As séries de registros da temperatura mínima do ar diária do mês de novembro monitorada nas cinco estações da região da UHE Baixo Iguaçu são apresentadas na
                   figura \ref{fig:figtemp_med_min_ar_11}.    
    

    \begin{figure}[!htb]
    \includegraphics[width=1 \textwidth]{temp_med_min_ar_11}
    \caption{Temperatura mínima do ar observada na região da UHE Baixo Iguaçu em novembro de 2020.}
    \label{fig:figtemp_med_min_ar_11}
    \end{figure}

    A partir das séries diárias do mês em questão foram calculados o valor médio da temperatura mínima do ar, assim como seus valores máximo, mínimo e
    o desvio padrão em cada estação. Os resultados são apresentados na tabela \ref{tab:tabtemp_med_min_ar_11}.
    
    \begin{table}[!hbt]
    \begin{center}
    \caption{Estatísticas da variável temperatura mínima do ar ($^\circ$C) no mês de novembro de 2020 nas estações meteorológicas da região.}
    \label{tab:tabtemp_med_min_ar_11}
    \begin{tabular}{lccccc}
    \hline
                  &        Cascavel        &        Dois Vizinhos        &        Planalto        &         \begin{tabular}[c]{@{}c@{}} São Miguel \\ do Iguaçu \end{tabular}        &        \begin{tabular}[c]{@{}c@{}} UHE Baixo \\ Iguaçu \end{tabular}        \\
    \hline
    Média         &        16.33     &        23.84     &        17.85     &        18.54     &        \textbf{16.20}     \\
    Máxima        &        21.20     &        32.20     &        23.10     &        24.60     &        \textbf{21.10}     \\
    Mínima        &        12.00     &        18.30     &        13.00     &        14.60     &        \textbf{9.80}     \\
    Desvio-padrão &        2.26     &        2.89     &        2.52     &        2.59     &        \textbf{2.85}     \\
    \hline          
    \end{tabular}
    \end{center}
    \end{table}
    


    
    \newpage
     
                   \subsection{Umidade relativa}
                   
                   \hspace{0.5cm} As séries de registros da umidade relativa diária do mês de novembro monitorada nas cinco estações da região da UHE Baixo Iguaçu são apresentadas na
                   figura \ref{fig:figumidade_relativa_11}.    
    

    \begin{figure}[!htb]
    \includegraphics[width=1 \textwidth]{umidade_relativa_11}
    \caption{Umidade relativa observada na região da UHE Baixo Iguaçu em novembro de 2020.}
    \label{fig:figumidade_relativa_11}
    \end{figure}

    A partir das séries diárias do mês em questão foram calculados o valor médio da umidade relativa, assim como seus valores máximo, mínimo e
    o desvio padrão em cada estação. Os resultados são apresentados na tabela \ref{tab:tabumidade_relativa_11}.
    
    \begin{table}[!hbt]
    \begin{center}
    \caption{Estatísticas da variável umidade relativa (\%) no mês de novembro de 2020 nas estações meteorológicas da região.}
    \label{tab:tabumidade_relativa_11}
    \begin{tabular}{lccccc}
    \hline
                  &        Cascavel        &        Dois Vizinhos        &        Planalto        &         \begin{tabular}[c]{@{}c@{}} São Miguel \\ do Iguaçu \end{tabular}        &        \begin{tabular}[c]{@{}c@{}} UHE Baixo \\ Iguaçu \end{tabular}        \\
    \hline
    Média         &        58.01     &        48.44     &        53.96     &        54.64     &        \textbf{65.04}     \\
    Máxima        &        83.10     &        99.25     &        80.29     &        82.50     &        \textbf{85.14}     \\
    Mínima        &        33.73     &        23.80     &        31.54     &        33.40     &        \textbf{51.17}     \\
    Desvio-padrão &        15.02     &        18.06     &        16.22     &        14.52     &        \textbf{10.25}     \\
    \hline          
    \end{tabular}
    \end{center}
    \end{table}
    


    
    \newpage
     
                   \subsection{Pressão atmosférica}
                   
                   \hspace{0.5cm} As séries de registros da pressão atmosférica diária do mês de novembro monitorada nas cinco estações da região da UHE Baixo Iguaçu são apresentadas na
                   figura \ref{fig:figpressao_atm_11}.    
    

    \begin{figure}[!htb]
    \includegraphics[width=1 \textwidth]{pressao_atm_11}
    \caption{Pressão atmosférica observada na região da UHE Baixo Iguaçu em novembro de 2020.}
    \label{fig:figpressao_atm_11}
    \end{figure}

    A partir das séries diárias do mês em questão foram calculados o valor médio da pressão atmosférica, assim como seus valores máximo, mínimo e
    o desvio padrão em cada estação. Os resultados são apresentados na tabela \ref{tab:tabpressao_atm_11}.
    
    \begin{table}[!hbt]
    \begin{center}
    \caption{Estatísticas da variável pressão atmosférica (hPa) no mês de novembro de 2020 nas estações meteorológicas da região.}
    \label{tab:tabpressao_atm_11}
    \begin{tabular}{lccccc}
    \hline
                  &        Cascavel        &        Dois Vizinhos        &        Planalto        &         \begin{tabular}[c]{@{}c@{}} São Miguel \\ do Iguaçu \end{tabular}        &        \begin{tabular}[c]{@{}c@{}} UHE Baixo \\ Iguaçu \end{tabular}        \\
    \hline
    Média         &        936.08     &        950.60     &        966.34     &        977.28     &        \textbf{979.86}     \\
    Máxima        &        938.64     &        953.64     &        968.79     &        980.03     &        \textbf{982.43}     \\
    Mínima        &        932.10     &        946.15     &        962.15     &        972.72     &        \textbf{975.41}     \\
    Desvio-padrão &        1.68     &        1.85     &        1.63     &        1.61     &        \textbf{1.68}     \\
    \hline          
    \end{tabular}
    \end{center}
    \end{table}
    


    
    \newpage
     
                   \subsection{Radiação solar incidente}
                   
                   \hspace{0.5cm} As séries de registros da radiação solar incidente diária do mês de novembro monitorada nas cinco estações da região da UHE Baixo Iguaçu são apresentadas na
                   figura \ref{fig:figradiacao_11}.    
    

    \begin{figure}[!htb]
    \includegraphics[width=1 \textwidth]{radiacao_11}
    \caption{Radiação solar incidente observada na região da UHE Baixo Iguaçu em novembro de 2020.}
    \label{fig:figradiacao_11}
    \end{figure}

    A partir das séries diárias do mês em questão foram calculados o valor médio da radiação solar incidente, assim como seus valores máximo, mínimo e
    o desvio padrão em cada estação. Os resultados são apresentados na tabela \ref{tab:tabradiacao_11}.
    
    \begin{table}[!hbt]
    \begin{center}
    \caption{Estatísticas da variável radiação solar incidente (MJ m$^{-2}$ dia$^{-1}$) no mês de novembro de 2020 nas estações meteorológicas da região.}
    \label{tab:tabradiacao_11}
    \begin{tabular}{lccccc}
    \hline
                  &        Cascavel        &        Dois Vizinhos        &        Planalto        &         \begin{tabular}[c]{@{}c@{}} São Miguel \\ do Iguaçu \end{tabular}        &        \begin{tabular}[c]{@{}c@{}} UHE Baixo \\ Iguaçu \end{tabular}        \\
    \hline
    Média         &        21.39     &        ---     &        23.06     &        21.54     &        \textbf{22.97}     \\
    Máxima        &        31.43     &        ---     &        31.49     &        28.71     &        \textbf{31.28}     \\
    Mínima        &        4.96     &        ---     &        5.58     &        6.10     &        \textbf{5.29}     \\
    Desvio-padrão &        7.49     &        ---     &        7.76     &        6.98     &        \textbf{8.05}     \\
    \hline          
    \end{tabular}
    \end{center}
    \end{table}
    


    
    \newpage
     
                   \subsection{Evapotranspiração }
                   
                   \hspace{0.5cm} Uma vez que não há medição direta de evapotranspiração na região monitorada, optou-se por calcular a
                   evapotranspiração de referência com base no método de Penman-Monteith sugerido pela 
                   Organização das Nações Unidas para Agricultura e Alimentação (FAO). 
                   
                   Na sequência, a figura \ref{fig:figevapo_11} exibe os valores diários de evapotranspiração ao longo do mês de novembro nas cinco estações
                   monitoradas na área de interesse.
                       
    

    \begin{figure}[!htb]
    \includegraphics[width=1 \textwidth]{evapo_11}
    \caption{Evapotranspiração observada na região da UHE Baixo Iguaçu em novembro de 2020.}
    \label{fig:figevapo_11}
    \end{figure}

    A partir das séries diárias do mês em questão foram calculados o valor médio da evapotranspiração, assim como seus valores máximo, mínimo e
    o desvio padrão em cada estação. Os resultados são apresentados na tabela \ref{tab:tabevapo_11}.
    
    \begin{table}[!hbt]
    \begin{center}
    \caption{Estatísticas da variável evapotranspiração (mm/dia) no mês de novembro de 2020 nas estações meteorológicas da região.}
    \label{tab:tabevapo_11}
    \begin{tabular}{lccccc}
    \hline
                  &        Cascavel        &        Dois Vizinhos        &        Planalto        &         \begin{tabular}[c]{@{}c@{}} São Miguel \\ do Iguaçu \end{tabular}        &        \begin{tabular}[c]{@{}c@{}} UHE Baixo \\ Iguaçu \end{tabular}        \\
    \hline
    Média         &        5.99     &        ---     &        5.87     &        5.66     &        \textbf{5.31}     \\
    Máxima        &        8.32     &        ---     &        8.20     &        9.14     &        \textbf{7.60}     \\
    Mínima        &        2.22     &        ---     &        2.15     &        2.07     &        \textbf{1.62}     \\
    Desvio-padrão &        1.74     &        ---     &        1.82     &        1.74     &        \textbf{1.54}     \\
    \hline          
    \end{tabular}
    \end{center}
    \end{table}
    


    
    \newpage
     
                   \subsection{Velocidade}
                   
                   \hspace{0.5cm} As séries de registros da velocidade diária do mês de novembro monitorada nas cinco estações da região da UHE Baixo Iguaçu são apresentadas na
                   figura \ref{fig:figvelocidades_11}.    
    

    \begin{figure}[!htb]
    \includegraphics[width=1 \textwidth]{velocidades_11}
    \caption{Velocidade observada na região da UHE Baixo Iguaçu em novembro de 2020.}
    \label{fig:figvelocidades_11}
    \end{figure}

    A partir das séries diárias do mês em questão foram calculados o valor médio da velocidade, assim como seus valores máximo, mínimo e
    o desvio padrão em cada estação. Os resultados são apresentados na tabela \ref{tab:tabvelocidades_11}.
    
    \begin{table}[!hbt]
    \begin{center}
    \caption{Estatísticas da variável velocidade (m/s) no mês de novembro de 2020 nas estações meteorológicas da região.}
    \label{tab:tabvelocidades_11}
    \begin{tabular}{lccccc}
    \hline
                  &        Cascavel        &        Dois Vizinhos        &        Planalto        &         \begin{tabular}[c]{@{}c@{}} São Miguel \\ do Iguaçu \end{tabular}        &        \begin{tabular}[c]{@{}c@{}} UHE Baixo \\ Iguaçu \end{tabular}        \\
    \hline
    Média         &        4.10     &        1.41     &        2.35     &        2.44     &        \textbf{1.66}     \\
    Máxima        &        6.24     &        2.52     &        4.26     &        4.55     &        \textbf{3.09}     \\
    Mínima        &        2.51     &        0.04     &        0.17     &        1.47     &        \textbf{1.17}     \\
    Desvio-padrão &        1.02     &        0.72     &        0.94     &        0.75     &        \textbf{0.39}     \\
    \hline          
    \end{tabular}
    \end{center}
    \end{table}
    


    
    \newpage
      

                   \subsection{Direção do vento}
                   \hspace{0.5cm} A figura \ref{fig:figdir} exibe um diagrama com as direções predominantes da velocidade
                   do vento em cada dia dos mês. Além disso, para melhor visualização das condições do vento em cada estação, 
                   as linhas contínuas representam as séries de velocidade do vento normalizadas
                   pela velocidade máxima do mês registrada em cada estação.
    
    \begin{figure}[!htb]
    \includegraphics[width=1. \textwidth]{diagramaveldir_11}
    \caption{Diagrama da direção do vento predominante registrada nas estações meteorológicas na região da UHE Baixo Iguaçu.}
    \label{fig:figdir}
    \end{figure}
    
    A partir da série de dados de direção do vento no mês em questão foram contabilizadas as frequências em cada setor, conforme exibe a 
    tabela \ref{tab:tabdir}.

    \begin{table}[!hbt]
    \begin{center}
    \caption{Frequência (\%) das direções do vento medidos nas estações meteorológicas no mês de novembro de 2020.}
    \label{tab:tabdir}
    \begin{tabular}{lcccccccc}
    \hline
             &        N      &        NE     &        L      &        SE     &        S     &        SO     &        O    &        NO  \\
    \hline                                                                                                                            
    Cascavel       &        0.00     &        50.00     &        13.33     &        20.00     &        6.67    &        6.67     &        0.00   &        3.33   \\
    Dois Vizinhos       &        20.00     &        12.00     &        8.00     &        36.00     &        4.00    &        8.00     &        4.00   &        8.00   \\
    Planalto       &        6.67     &        3.33     &        6.67     &        0.00     &        0.00    &        70.00     &        13.33   &        0.00   \\
    São Miguel do Iguaçu       &        3.33     &        10.00     &        0.00     &        73.33     &        0.00    &        0.00     &        3.33   &        10.00   \\
    \textbf{UHE Baixo Iguacu}       &        \textbf{16.67}     &        \textbf{36.67}     &        \textbf{33.33}     &        \textbf{3.33}     &        \textbf{0.00}    &        \textbf{0.00}     &        \textbf{6.67}   &        \textbf{3.33}   \\

    \hline          
    \end{tabular}
    \end{center}
    \end{table}

\newpage


                   \subsection{Precipitação}
                   \hspace{0.5cm} As séries de registros de precipitação diária nas estações telemétricas da região da UHE Baixo Iguaçu são exibidas na figura \ref{fig:figchuva}, e a
                   tabela \ref{tab:tabchuva} exibe o acumulado mensal de cada estação.

    \begin{figure}[!htb]
    \includegraphics[width=1 \textwidth]{chuva_11}
    \caption{Registros de chuva acumulada diária observada na região da UHE Baixo Iguaçu ao longo de novembro de 2020.}
    \label{fig:figchuva}
    \end{figure}
    
    Na sequência, a tabela \ref{tab:tabchuva} exibe os valores da chuva acumulada mensal em cada estação, 
    assim como o valor do maior registro de chuva diária.
    
    \begin{table}[!htb]
    \centering
    \caption{Chuva acumulada (mm) nas estações telemétricas próximas à região da UHE Baixo Iguaçu ao longo de novembro de 2020.}
    \label{tab:tabchuva}
    \begin{tabular}{lcc}
    \hline
                           & Chuva diária máxima      &   Chuva acumulada mensal \\
    \hline
      Cascavel                   &  23.20                   &   71.40    \\
      Dois Vizinhos                   &  0.80                   &   1.20    \\
      Planalto                   &  9.00                   &   42.40    \\
      São Miguel do Iguaçu                   &  8.80                   &   30.20    \\
      \textbf{UHE Baixo Iguacu}                   &  \textbf{18.60}                   &   \textbf{73.20}     \\
    \hline
    \end{tabular}
    \end{table}
        

\end{document}